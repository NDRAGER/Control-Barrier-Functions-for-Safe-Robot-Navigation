\documentclass{beamer}
\usetheme{CambridgeUS}
\usecolortheme{default}
\usebeamerfont{structurebold}
\usepackage{graphicx}
\usepackage{amsmath}
\usepackage{amssymb}
\usepackage{hyperref}
\usepackage{tikz}
\usetikzlibrary{shapes,arrows,positioning}

\title{Efficient Safe Robot Navigation Using Control Barrier Functions}
\author{Nicolas Drager, Satvik Tajane, Harsh Akabari}

\setbeamertemplate{footline}
{
	\leavevmode%
	\hbox{%
		\begin{beamercolorbox}[wd=1.0\paperwidth,ht=2.25ex,dp=1ex,right]{date in head/foot}%
			\usebeamerfont{date in head/foot}
			\insertframenumber{} / \inserttotalframenumber\hspace*{2ex}
		\end{beamercolorbox}%
	}%
	\vskip0pt%
}

\begin{document}
	
	\begin{frame}
		\titlepage
	\end{frame}
	
	\begin{frame}{Outline}
		\tableofcontents
	\end{frame}
	
	% ========== INTRODUCTION ==========
	\section{Introduction}
	
	\begin{frame}{Problem: Safe Navigation in Dynamic Environments}
		\textbf{Safety is paramount in robotics}
		\begin{itemize}
			\item Primary concern: collision avoidance
			\item But robots must also reach their goals efficiently
		\end{itemize}
		\textbf{ \(\implies \) avoid collisions but keep moving}
		
		\vspace{0.3cm}
		\textbf{Application context:}
		\begin{itemize}
			\item Multi-robot warehouses: many robots, dynamic environment
			\item High-speed operations require real-time safety guarantees
		\end{itemize}
	\end{frame}
	
	\begin{frame}{Solution: CBF as a Safety Filter}
		\begin{block}{Core Idea}
			Control Barrier Functions act as a \textbf{safety filter}:
			\begin{itemize}
				\item Filters out unsafe commands
				\item Minimally modifies safe ones for efficient avoidance in the future
			\end{itemize}
		\end{block}
		
		\vspace{0.3cm}
		\textbf{Our Approach:}
		\begin{itemize}
			\item Mathematical safety guarantees via CBF constraints
			\item \textbf{Turn-first strategy}: prioritize turning over stopping
			\begin{itemize}
				\item Maintains forward momentum
				\item Faster navigation through tight spaces
			\end{itemize}
			\item Priority-based multi-robot coordination
		\end{itemize}
	\end{frame}
	
	% ========== ALGORITHM ==========
	\section{Algorithm}
	
	\begin{frame}{Control Barrier Functions: Core Concept}
		\begin{block}{Safety Definition}
			CBF $h(\mathbf{x})$ defines safe region: $\mathcal{C} = \{\mathbf{x} \mid h(\mathbf{x}) \geq 0\}$
		\end{block}
		
		\begin{block}{Safety Constraint}
			If control $\mathbf{u}$ satisfies:
			\begin{equation}
				\dot{h}(\mathbf{x}, \mathbf{u}) + \gamma h(\mathbf{x}) \geq 0, \quad \gamma > 0
			\end{equation}
			then system remains safe: $h(\mathbf{x}) \geq 0$ for all time
		\end{block}
		
	\end{frame}
	
	\begin{frame}{Simulated Robot Dynamics \& Constraints}
		\textbf{Differential drive robot modeled as unicycle:}
		\begin{equation}
			\dot{x} = v \cos\theta, \quad \dot{y} = v \sin\theta, \quad \dot{\theta} = \omega
		\end{equation}
		
		\textbf{Control Inputs:}
		\begin{itemize}
			\item Linear velocity: $v$
			\item Angular velocity: $\omega$
		\end{itemize}

		\textbf{Physical limits:}
		\begin{itemize}
			\item Velocity: $v \in [0, 0.5]$ m/s, $\omega \in [-1.5, 1.5]$ rad/s
			\item Acceleration: $|\dot{v}| \leq 0.5$ m/s$^2$, $|\dot{\omega}| \leq 2.0$ rad/s$^2$
			\item Robot radius: $R = 0.2$ m
		\end{itemize}

	\end{frame}
	
\begin{frame}{LIDAR-Based Surface Detection and Distance Measurement}
	
	\begin{columns}
		
		% Left column for text content
		\begin{column}{0.45\textwidth}
			\textbf{LIDAR} detects \textit{obstacle surfaces}
			
			\bigskip
			
			\begin{itemize}
				\item 360 rays at 1° resolution to detect surfaces and their distance to robot
				\item Clustering (optional): group detections to identify objects
			\end{itemize}
		\end{column}
		
		% Right column for image
		\begin{column}{0.45\textwidth}
			\begin{figure}[h]
				\centering
				\includegraphics[width=\textwidth]{lidar.png}
			\end{figure}
		\end{column}
		
	\end{columns}
	
\end{frame}
	
		\begin{frame}{Barrier Function}
		\textbf{For closest detected surface point} $\mathbf{p}_{\text{obs}}$:
		\begin{block}{Barrier Function}
			\begin{equation}
				h(\mathbf{x}) = \|\mathbf{p}_{\text{robot}} - \mathbf{p}_{\text{obs}}\|^2 - (R_{\text{robot}} + \text{buffer})^2
			\end{equation}
		\end{block}
		
		\(\implies\)\textbf{Safety buffer} differs dependent on the detected object!
		
		\(\implies\)Allows the prioritization of objects, in our case fellow robots
	\end{frame}
	

	
	
	\begin{frame}{QP-Based Safe Controller}
		\textbf{Optimization problem:}
		\begin{equation}
			\begin{aligned}
				\min_{v, \omega} \quad & \|u - u_{\text{des}}\|^2 \\
				\text{s.t.} \quad & \dot{h} + \gamma h \geq 0 \\
				& v \in [v_{\text{prev}} - a_{\max}dt, v_{\text{prev}} + a_{\max}dt] \\
				& \omega \in [\omega_{\text{prev}} - \alpha_{\max}dt, \omega_{\text{prev}} + \alpha_{\max}dt]
			\end{aligned}
		\end{equation}
		
		\textbf{Notice:}
		\begin{itemize}
			\item Finds control closest to desired while guaranteeing safety
			\item Acceleration limits embedded in QP bounds (not post-processed)
			\item $\gamma = 2.0$ balances safety and agility
		\end{itemize}
	\end{frame}
	
	\begin{frame}{Prioritizing Turning Over Stopping}
		\textbf{Standard case} (not heading directly toward obstacle):
		\begin{equation}
			J(\mathbf{u}) = (v - v_{\text{des}})^2 + 0.5(\omega - \omega_{\text{des}})^2
		\end{equation}
		
		\textbf{Obstacle avoidance mode} (when $|\text{angle\_diff}| < 60°$ and $h < 1.0$):
		\begin{equation}
			J(\mathbf{u}) = \begin{cases}
				8.0 v^2 + 0.5(\omega - \omega_{\text{target}})^2 & \text{if } h < 0.3 \\
				4.0 v^2 + 0.5(\omega - \omega_{\text{target}})^2 & \text{if } 0.3 \leq h < 0.6 \\
				2.0(v - v_{\text{des}})^2 + 0.5(\omega - \omega_{\text{target}})^2 & \text{if } h \geq 0.6
			\end{cases}
		\end{equation}
		
		where $\omega_{\text{target}}$ encourages turning away from obstacle.
		%\vspace{0.3cm}
		%\textbf{Effect:}
		%\begin{itemize}
			%\item Encourages turning ($\omega_{\text{away}}$) to avoid obstacle
			%\item Reduces speed only when very close
			%\item Maintains momentum for efficient navigation
		%\end{itemize}
	\end{frame}
	
	\begin{frame}{Emergency Recovery}
		\textbf{When QP fails} (constraint infeasible):
		\begin{enumerate}
			\item Test $\dot{h}$ for left/right turns
			\item Choose direction maximizing $\dot{h}$
			\item Try $v = 0.05$ m/s + turn (if safe)
			\item Else: pure rotation $v = 0$
		\end{enumerate}
		
		\vspace{0.3cm}
		\textbf{Key:} Intelligent recovery maintains progress vs. static stopping
	\end{frame}
	
	\begin{frame}{Multi-Robot Priority System}
		\textbf{Problem:} Equal priorities $\rightarrow$ both robots stop/deadlock
		
		\textbf{Solution:} Dynamic priority based on context
		\begin{itemize}
			\item Robot closer to static obstacle gets priority
			\item Priority robot: buffer = $-0.05$ m for other robots
			\item Non-priority robot: buffer = $0$ m for other robots
		\end{itemize}
		
		\vspace{0.3cm}
		\textbf{Result:} Priority robot can "push" through, non-priority yields
	\end{frame}
	

	
	% ========== RESULTS ==========
	\section{Experimental Results}
	
		\begin{frame}{Scenario: One Robot with Obstacle}
		
		\begin{figure}[h]
			\centering
			\includegraphics[width=0.90\textwidth]{rectangle.png}
		\end{figure}
		\textbf{Observation:} Obstacle is detected and avoided by turning!
	\end{frame}
	
	\begin{frame}{Scenario: One Robot with Two Obstacles}
		
		\begin{figure}[h]
			\centering
			\includegraphics[width=0.90\textwidth]{2obstacle.png}
		\end{figure}
		\textbf{Observation:} Robot slows down and passes
	\end{frame}
	
	
	\begin{frame}{Scenario: Two Robots Crossover}
	
	\begin{figure}[h]
		\centering
		\includegraphics[width=0.90\textwidth]{cross.png}
	\end{figure}
	\(\implies\) One went in front of the other causing a emergency stop!
	\end{frame}
	
	
		\begin{frame}{Scenario: Two Robots with Obstacle (Priority System)}
		
		\begin{figure}[h]
			\centering
			\includegraphics[width=0.90\textwidth]{2robotobstacle.png}
		\end{figure}
	\(\implies\) The robot closest to the obstacle is the dominant one!
	\end{frame}
	

	


	% ========== CONCLUSION ==========
	\section{Conclusion}
	
	\begin{frame}{Contributions}
		\begin{itemize}
			\item \textbf{Turn-first strategy}: Maintains momentum while ensuring safety
			\item \textbf{Priority system}: Resolves multi-robot deadlocks
			\item \textbf{Intelligent recovery}: Handles QP failures gracefully
		\end{itemize}
	\end{frame}
	
	\begin{frame}{Potential Future Work}
		\textbf{Algorithmic improvements:}
		\begin{itemize}
			\item Velocity-dependent $\gamma$: more conservative at high speeds
			\item Multi-point CBF: constrain multiple surface points (prevents corner-cutting)
			\item Predictive CBF: account for changing closest point during motion
		\end{itemize}
		
		\vspace{0.3cm}
		\textbf{System extensions:}
		\begin{itemize}
			\item Scale to $N > 2$ robots with decentralized coordination
			\item Integration with global path planning
			\item Hardware validation on physical platforms
		\end{itemize}
	\end{frame}
	

	
\end{document}